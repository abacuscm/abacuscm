\documentclass[a4paper]{article}
\usepackage{a4wide}
\title{Abacus contestant manual}
\author{Bruce Merry}
\date{11 October 2006}
\begin{document}
\maketitle
\section{Introduction}
Abacus is the competition software used for the Southern African
Regional of the ACM International Collegiate Programming Challenge. You
will use it to submit your solutions and get feedback from the judges.

\section{Starting abacus}
The command to start abacus will depend on your local installation. In
most cases you should just be able to type \texttt{abacus} in a
terminal window.

After starting abacus, you should select \texttt{Connect} from the
\texttt{Abacus} menu. This will open a dialog box. The server and port
should have already been set for you. Fill in the username and password
that you can been given. If you do this correctly, you should receive
the message \texttt{You are now connected to the server.}

When the contest is running, you will see a clock in the bottom-right
corner counting down the remaining time.

\section{Clarifications}
If you are uncertain about some details in a task or the contest in
general, you may submit a clarification request. On the
\texttt{Clarification requests} tab, click the \texttt{Request
Clarification} button. Select the problem your query pertains to and
type in your query.

When a judge responds to your query, the response will pop up on your
screen. It will also be available on the clarifications tab (you can
double-click on clarification requests and clarifications to see the
full text).

Some things to take note of:
\begin{enumerate}
\item The newest information appears at the top.
\item The judges are located at the primary site. If you have questions
specifically about the installation at your site (e.g., your web
browser won't open) then speak to somebody at the site.
\item The judges do not know who is submitting queries. This means
that if you want to refer to a previous query you made, you should
include the request ID that appears in the list box. Similarly, if your
query is about a particular submission, you should include the
submission ID from the submissions tab.
\item You may receive clarifications for questions your never asked.
These are broadcast replies and indicate important information for all
contestants.
\end{enumerate}

\section{Writing your solution}
If you are using Java, you can call your main class whatever you
like, but it must be the \emph{first} class in the file. You can
include support classes after it, but they should not be
\texttt{public}.

Your solution will read from standard input
(\texttt{stdin}/\texttt{cin}/\texttt{System.in}) and write to standard
output (\texttt{stdout}/\texttt{cout}/\texttt{System.out}). Your
solution is tested by an automated marker, so you must be careful to
input only what is asked for, and output only what is asked for. In
particular, do not print prompts or messages like ``The answer is:''
unless you are asked to.

\section{Submitting}
When you are ready to submit a solution, go to the submissions page and
click \texttt{Make Submission}. Be very careful to select the right
problem, source file and language in the dialog box, as there are
penalty points for incorrect submissions and there is no way to reverse
an accidentally submitted solution.

Once you make a submission, it will appear in the list box, with newest
submissions at the top. The status field will let you keep track of the
submission status. The possible options are:
\begin{description}
\item[Pending] The submission is still in the automatic marker.
\item[Compilation failure] Your submission did not compile. This can
happen with C++ in particular, because the marker might not
be using the same version of the compiler. To assist you, you can
double-click on the submission to get the compiler output. In many
cases you just need to add a \verb"#include" statement. There is no
time penalty for compilation failures.
\item[Time limit exceeded] Your program exceeding the maximum time
limit. This could mean that it is too slow or has an infinite loop.
\item[Abnormal termination of program] Your program has crashed for
some reason. Possible reasons include using too much memory, writing an
excessive amount of output, throwing an uncaught Java exception, using
illegal memory in C/C++, and not returning zero as the exit code from
C/C++.
\item[Awaiting judge] The program produced an output file that was
judged wrong by the automatic marker. A judge will look at it, and
if the only error is in formatting it may be judged correct. You should
not rely on this.
\item[Wrong answer] A judge has confirmed that the answer is wrong.
\item[Correct answer] Well done!
\end{description}

When the status changes, you will be informed by a popup, so you do not
need to keep the submissions tab open.

\section{Standings}
You can see the standings at any time (from the standings tab), but
they will not be updated during the last hour of the contest. Only
teams that have made at least one submission will appear on the
standings.

\end{document}
